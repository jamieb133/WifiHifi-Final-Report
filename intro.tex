\providecommand{\main}{..}
\documentclass[main.tex]{subfiles}

\begin{document}
\chapter{Introduction}
\section{Background}
In recent years, the domain of consumer audio electronics has seen a significant increase in the utilisation of wireless technology as a transmission medium for audio signals from sources to loudspeakers.
Prominent examples of such systems include: wireless headphones and earbuds, portable bluetooth speakers and "smart" home speakers. 
In a home listening context, advantages of these systems over traditional wired solutions include the reduction of cabling, no reliance on a large power amplifier, and flexibility of speaker placement.

\medskip
A significant limitation of many existing solutions is the lack of spatial separation in the individual audio channels, this is mainly due to a single traducer being used which prevents the output of stereo audio.
While many wireless speakers contain a stereo pair of transducers, they are often housed in the same enclosure and therefore separation is limited due to the fixed positioning of each driver.
This also disables the ability to alter the speaker positioning within a room unlike a traditional HiFi system in which each loudspeaker cabinet can be moved to provide a preferred acoustic response for a given room. 

\medskip
Furthermore, existing solutions that do enable multi-channel audio through the use of separate loudspeaker cabinets are usually limited to a simple stereo-pair or occasionally a standard surround sound configuration such as 5.1 or 7.1 for home cinema systems.
There is therefore no ability to route a custom configuration of audio channels to a given set of speakers as the user is constrained to the configurations implemented by the manufacturer's proprietary software controls. 
This project therefore aims to create a fully operational multi-channel audio system that allows for each audio stream to be routed in any configuration to one, or many speakers.


\section{Current Solutions}
\subsection{RF Tranceivers}
RF communication can be used to send audio wirelessly, however this would require specialised equipment, and ideally our speakers could be controlled through computers or phones with no extra peripherals required.
Because of this it is not suitable for our desired application.
\subsection{Bluetooth}
Bluetooth is commonly used for wireless audio transmission for personal wireless headphones and small speakers.
However, due to Bluetooth's low bandwidth, audio has to be compressed in order to be transmitted.
Also, its range is limited, which would present a problem for home hi-fi use.
As such, it was not considered for use in this project.
\subsection{WiFi}
WiFi is the communication we chose to use for our project, as it has high bandwidth and speed, and is usable from any phone or computer.
Quality or range would not be a problem.

WiFi is currently used in commercial audio streaming applications like Sonos speaker systems.
These systems are very closed off, and such they are largely inflexible in many use cases.
In our project, our system will be open and flexible to any use case and number of speakers.

\section{Project Aims}
The ideal outcome of this project are a system that can be used for any arbitrary number of speakers and controlled from any computer or phone.
The main challenge of this was understood to be the process of synchronisation between speakers wirelessly, while maintaining good, hi-fi quality sound.

\subsection{Use Cases}
The specific speaker system we will be designing will be a reasonably good quality home hi-fi system, where the audio can be streamed from any home computer or phone.
The software that would run on the speaker would be able to be adapted to any system and and context with arbitrary number of channels.
\subsection{Requirements}
\end{document}