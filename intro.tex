\providecommand{\main}{..}
\documentclass[main.tex]{subfiles}

\begin{document}
\chapter{Introduction}
\section{Background}
In recent years, the domain of consumer audio electronics has seen a significant increase in the utilisation of wireless technology as a transmission medium for audio signals from sources to loudspeakers.
Prominent examples of such systems include: wireless headphones and earbuds, portable bluetooth speakers and "smart" home speakers. 
In a home listening context, advantages of these systems over traditional wired solutions include the reduction of cabling, no reliance on a large power amplifier, and flexibility of speaker placement.

\medskip
A significant limitation of many existing solutions is the lack of spatial separation in the individual audio channels, this is mainly due to a single traducer being used which prevents the output of stereo audio.
While many wireless speakers contain a stereo pair of transducers, they are often housed in the same enclosure and therefore separation is limited due to the fixed positioning of each driver.
This also disables the ability to alter the speaker positioning within a room unlike a traditional HiFi system in which each loudspeaker cabinet can be moved to provide a preferred acoustic response for a given room. 

\medskip
Furthermore, existing solutions that do enable multi-channel audio through the use of separate loudspeaker cabinets are usually limited to a simple stereo-pair or occasionally a standard surround sound configuration such as 5.1 or 7.1 for home cinema systems.
There is therefore no ability to route a custom configuration of audio channels to a given set of speakers as the user is constrained to the configurations implemented by the manufacturer's proprietary software controls. 
This project therefore aims to create a fully operational multi-channel audio system that allows for each audio stream to be routed in any configuration to one, or many speakers.


\section{Current Solutions}
\subsection{RF Tranceivers}
\subsection{Bluetooth}
\subsection{WiFi}

\section{Project Aims}
\subsection{Use Cases}
\subsection{Requirements}
The general project requirements are outlined as follows:

\begin{itemize}
    \item Should have a professional quality design.
    \item Will be a standard bookshelf design.
    \item Audio properties such as tone control and volume to be controlled via a remote app.
    \item Should be platform agnostic so that app can be used on multiple operating systems.
    \item Channel configuration should be controllable for all speakers.
    \item Sample rate must be a minimum of 44.1kHz.
    \item Bit depth must be a minimum of 16-bit.
    \item Individual speakers should output with an average phase error of approximately 5ms.
\end{itemize}
\end{document}